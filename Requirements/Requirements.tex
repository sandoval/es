% !TEX encoding = UTF-8 Unicode
\documentclass[12pt,a4paper]{report}
 
\usepackage[brazil]{babel}
\usepackage[utf8]{inputenc}
\usepackage[T1]{fontenc}
\usepackage{graphicx, subfigure}
\usepackage{indentfirst}

\graphicspath{ {images/} }

\title{Especificação de Requisitos do Sistema de Gestão Eletrônica de Documentos da Defensoria Pública do Distrito Federal - SGED/DPDF}
\author{Pedro Salum\\
	09/0139232\\
	pedro@loopec.com.br
	\and
	Daniel Sandoval\\
	09/0109899\\
	daniel@loopec.com.br}
	
\begin{document}
\maketitle
\tableofcontents

\chapter{Introdução}

Este documento apresenta os requisitos levantados para o Sistema de Gestão Eletrônica de Documentos da Defensoria Pública do Distrito Federal - SGED/DPDF. Os requisitos levantados foram divididos em quatro categorias: \textit{user stories}, requisitos não funcionais, requisitos de sistema e requisitos de domínio.

\paragraph{\textit{User stories}} Devido à utilização de métodos ágeis para o desenvolvimento, os requisitos de usuário foram levantados já em formato de \textit{user stories}: ações que cada usuário gostaria de poder realizar no sistema, seguindo o modelo <realizar ação> para <atingir objetivo>, para cada perfil de usuário identificado no sistema.

\paragraph{Requisitos não funcionais} São requisitos que não representam uma funcionalidade necessária para se realizar uma ação no sistema propriamente dita, mas que dizem respeito ao seu funcionamento geral.

\paragraph{Requisitos de sistema} São requisitos diretamente relacionados à existência do sistema em si e do ambiente criado e gerenciado por ele.

\paragraph{Requisitos de domínio} São requisitos identificados levando-se em consideração o domínio da aplicação, como restrições legais, restrições de órgãos reguladores ou restrições intrínsecas ao tipo de negócio e suas entidades.

\chapter{User Stories}

Cada seção deste capítulo enumera as \textit{user stories} de um papel de usuário no sistema.

\section{Diretor}
\begin{itemize}
\item[-] Quero obter estatísticas sobre os processos para avaliar a capacidade produtiva da DP-DF
\end{itemize}

\section{Defensor ou Colaborador}
\begin{itemize}
\item[-] Quero modificar peças escritas pelos estagiários para poder corrigir eventuais erros.
\item[-] Quero consultar um processo pelo seu número para saber o seu status
\item[-] Quero reivindicar a posse de um documento para atualizar a sua custódia.
\item[-] Quero escanear os documentos referentes ao processo, solicitados ou não pelo juiz, para complementar o processo
\item[-] Quero escanear os documentos (ID, CPF, Endereço…) para complementar o cadastro de um assistido.
\item[-] Quero acessar uma peça para colaborar no desenvolvimento de um documento.
\item[-] Quero inserir dados dos assistidos para poder auxiliar na geração automática de documentos.
\end{itemize}

\section{Servidor}
\begin{itemize}
\item[-] Quero registrar a entrada de documentos na DP-DF provenientes de Tribunais/Forums para iniciar o processo de controle de custódia.
\item[-] Quero registrar a saída de documentos da DP-DF para os Tribunais para atualizar o controle de custódia
\item[-] Quero pesquisar o documento pelo código para saber o seu histórico.
\end{itemize}

\section{Estagiário}
\begin{itemize}
\item[-] Quero escanear os documentos (ID, CPF, Endereço…) para complementar o cadastro de um assistido.
\item[-] Quero acessar uma peça para colaborar no desenvolvimento de um documento.
\item[-] Quero inserir dados dos assistidos para poder auxiliar na geração automática de documentos.
\end{itemize}

\section{Usuário}
\begin{itemize}
\item[-] Quero inserir minha matrícula e senha para iniciar uma sessão autenticada no sistema.
\item[-] Quero clicar no botão para finalizar a minha sessão no sistema.
\end{itemize}

\section{Administrador}
\begin{itemize}
\item[-] Quero alterar as permissões para controlar o nível de acesso dos usuários
\end{itemize}
%REQUISITOS NÃO FUNCIONAIS
%	- O sistema deve ser capaz de efetuar integração com scanners.
%	- O sistema deve ser capaz de se manter disponível mesmo se a conexão com o servidor principal estiver instável.
%	- O sistema deve se comunicar com os outros nós de forma segura e criptografada
%	- O sistema deve disponibilizar ícones com duas densidades de pixel: 1;1.45;
%	- O sistema não deve permitir acesso as funcionalidades para usuários não autenticados
%	- O sistema não deve ser acessível fora da GDFNET
%	- O sistema deve estar disponível 24 horas por dia/7 dias por semana.
%	- O sistema deve estar disponível na língua PT_BR
%	- O sistema deve autenticar o usuário em, no máximo, 3 segundos. Considerando que as configurações mínimas de hardware foram atendidas.
%	- 
%REQUISITO DE SISTEMA
%	- O sistema deve associar todo e qualquer documento pessoal escaneado à um assistido.
%	- O sistema não deve permitir duas sessões ativas de um mesmo usuário
%
%REQUISITO DE DOMINIO
%	- O sistema não deve permitir que uma peça já finalizada e enviada seja alterada
%	- O sistema não deve permitir que um Defensor/Colaborador reivindique a posse de um documento que já está sob sua custódia
%	- O sistema não deve permitir o anexo de novos documentos em um processo extinto
%	- O sistema não deve permitir o registro duplicado de documentos
%	- O sistema não deve permitir o registro de entrada de um documento que já está sob a custódia da DP-DF
%	- O sistema não deve permitir o registro de saída de um documento que não está sob a custódia.
%	- O sistema não deve permitir a associação de mais de um documento de mesma natureza ao mesmo assistidos (2 cpfs)
%	- O sistema não deve permitir que usuários de nível inferior altere permissões de usuários superiores na hierarquia.
%	- O código de um documento é único e definido por um conjunto de 10 caracteres alfanuméricos
%	- A senha deve conter no mínimo 5 caracteres alfanuméricos, com ao menos um número e um caractere especial.

\end{document}
