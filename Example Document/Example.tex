% !TEX encoding = UTF-8 Unicode
\documentclass[12pt,a4paper]{report}
 
\usepackage[brazil]{babel}
\usepackage[utf8]{inputenc}
\usepackage[T1]{fontenc}
\usepackage{graphicx, subfigure}
\usepackage{indentfirst}

\graphicspath{ {images/} }

\title{Modelo de Relatório}
\author{Pedro Salum\\
	09/0139232\\
	pedro@loopec.com.br
	\and
	Daniel Sandoval\\
	09/0109899\\
	daniel@loopec.com.br}
	
\begin{document}
\maketitle
\tableofcontents

\chapter{Capítulo}

\section{Seção}

%Colocando figura...
%\begin{figure}[h]
%\centering
%\includegraphics[width=0.7\textwidth]{arduino.png}
%\caption{Arduino Uno}
%\label{fig:arduino}
%\end{figure}

%\begin{enumerate}
%\item [1.] Após a realização das 800 leituras, o Arduino fica em estado de espera;
%\item [2.] O computador envia um caracter ao Arduino, solicitando uma leitura;
%\item [3.] O Arduino envia ao computador a próxima leitura;
%\item [4.] Os itens 2 e 3 se repetem até que sejam transmitidas as 800 leituras.
%\end{enumerate}

%\begin{equation}
%f (t) = \frac{a_{0}}{2} + \sum_{n=1}^{\infty} [ a_{n}\cos{(\frac{n \pi t}{L})} + b_{n}\sin{(\frac{n \pi t}{L})} ]
%\label{eq:serie_fourier}
%\end{equation}

%\begin{figure}[ht!]
%     \begin{center}
%        \subfigure[]{
%            \label{fig:onda_seno}
%            \includegraphics[width=0.47\textwidth]{seno.png}
%        }
%        \subfigure[]{
%           \label{fig:onda_seno_fft}
%           \includegraphics[width=0.47\textwidth]{seno_fft.png}
%        }\\
%        \subfigure[]{
%            \label{fig:onda_triangular}
%            \includegraphics[width=0.47\textwidth]{triangular.png}
%        }
%        \subfigure[]{
%            \label{fig:onda_triangular_fft}
%            \includegraphics[width=0.47\textwidth]{triangular_fft.png}
%        }\\
%        \subfigure[]{
%            \label{fig:onda_quadrada}
%            \includegraphics[width=0.47\textwidth]{quadrada.png}
%        }
%        \subfigure[]{
%            \label{fig:onda_quadrada_fft}
%            \includegraphics[width=0.47\textwidth]{quadrada_fft.png}
%        }
%    \end{center}
%    \caption{
%     	Sinais adquiridos através da solução apresentada: (a) onda senoidal; (c) onda triangular; (e) onda quadrada; e Análise no espectro da frequência dos sinais obtidos: (b) análise do sinal senoidal, (d) análise do sinal triangular; (f) análise do sinal quadrado.
%     }
%   \label{fig:resultados}
%\end{figure}
%
%\begin{thebibliography}{9}
%
%\bibitem{integrationbyexample}
%  Boesch, F.,
%  \emph{Integration by Example - Euler vs Verlet vs Runge-Kutta}.
%  \textit{http://codeflow.org/entries/2010/aug/28/integration-by-example-euler-vs-verlet-vs-runge-kutta/}
%  
%\bibitem{calculus}
%  Leithold, L.; , The Calculus with Analytic Geometry, 6a Ed., HarperCollins Publishers, 1990.
%
%\bibitem{atmega328}
%	ATmega328 \textit{http://www.atmel.com/devices/atmega328.aspx}
%	
%\bibitem{analogdigital}
%	Analog and Digital \textit{http://www.stanford.edu/class/cs101/analog-digital.html}
%	
%\bibitem{transducers}
%	What are Transducers? \textit{http://www.wisegeek.org/what-are-transducers.htm}
%	
%\bibitem{dsp}
%	Processamento Digital de Sinais - Vantagens e Aplicações \textit{http://pt.scribd.com/doc/49815347/Processamento-Digital-de-Sinais-Vantagens-e-Aplicacoes}
%	
%\end{thebibliography}

\end{document}
