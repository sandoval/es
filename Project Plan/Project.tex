% !TEX encoding = UTF-8 Unicode
\documentclass[12pt,a4paper]{report}
 
\usepackage[brazil]{babel}
\usepackage[utf8]{inputenc}
\usepackage[T1]{fontenc}
\usepackage{graphicx, subfigure}
\usepackage{epstopdf}
\usepackage{indentfirst}
\usepackage{fancyhdr}

\graphicspath{ {../images/} }

\epstopdfsetup{outdir=./}
\epstopdfsetup{suffix=}

\pagestyle{fancy}

\fancyhead[RO,RE]{\includegraphics{unb.eps}}
\fancyhead[LO,LE]{Departamento de Ciência da Computação\\
116441 --- Engenharia de Software}

\fancypagestyle{plain}{
\fancyhead[RO,RE]{\includegraphics{unb.eps}}
\fancyhead[LO,LE]{Departamento de Ciência da Computação\\
116441 --- Engenharia de Software}}

\title{Plano de Projeto do Sistema de Gestão Eletrônica de Documentos da Defensoria Pública do Distrito Federal - SGED/DPDF}
\author{Pedro Salum\\
	09/0139232\\
	pedro@loopec.com.br
	\and
	Daniel Sandoval\\
	09/0109899\\
	daniel@loopec.com.br}
	
\begin{document}
\maketitle
\tableofcontents

\chapter{Introdução}

Este documento tem por objetivo explicitar o planejamento para a realização do projeto de Gestão Eletrônica de Documentos da Defensoria Pública do Distrito Federal --- SGED/DPDF --- no tocante ao desenvolvimento, arquitetura, testes, implantação, manutenção e riscos. A seguir há uma breve descrição de cada capítulo.

\paragraph{Processo de Desenvolvimento} Discorre sobre decisões tomadas em relação a metodologias de desenvolvimento de \textit{software} e gerência de projeto. Discute a escolha de metodologias ágeis sobre métodos planejados.

\paragraph{Arquitetura e padrões de projeto} Discorre sobre escolhas de arquitetura e padrões de projeto que serão utilizados na codificação do sistema.

\paragraph{Plano de Testes} Discorre sobre as metodologias e ferramentas de testes escolhidas para o projeto, considerando deste testes de código até de aceitação, como e quando serão realizados.

\paragraph{Plano de Implantação e Manutenção} Discorre sobre as metodologias e ferramentas escolhidas para o processo de implantação e manutenção contínua do sistema entregue.

\paragraph{Análise de Riscos} Discorre sobre riscos que devem ser considerados ao longo do projeto e possibilidades de mitigação.

\chapter{Processo de Desenvolvimento}

A metodologia de desenvolvimento escolhida foi baseada em métodos ágeis. Levando-se em consideração que o sistema a ser desenvolvido está inserido em um ambiente corporativo, temos que estar preparados para mudanças. Mudanças de prioridades; regras de negócio; gerência; cronograma; entre outras, devem ser suportadas com o mínimo de prejuízo possível. Além disso, há experiências bem sucedidas com desenvolvimento ágil em ambientes corporativos governamentais \cite{metaginstpub}.

Sabendo que métodos ágeis não são próprios para implantação uniforme em qualquer situação \cite{cohn2009succeeding}, as ferramentas e metodologias escolhidas para este projeto são uma sugestão e podem sofrer modificações ou adaptações ao longo do desenvolvimento.

\paragraph{Scrum} 
%-> Definição e explicação de qual processo de desenvolvimento de software escolhido para o projeto

\chapter{Arquitetura e padrões de projeto}

%-> Definição e explicação da arquitetura de software escolhida para o projeto, focalizando a escolha de padrões de arquitetura, assim como diagrama de arquitetura
%-> Definição e explicação dos design patterns escolhidos para serem implementados no projeto

\chapter{Plano de Testes}
%-> Discussão dos testes de verificação e validação realizados no projeto, assim como descrição de casos de testes, inspeção de código, testes unitários

\chapter{Plano de Implantação e Manutenção}
%-> Discussão da estratégia de implantação e manutenção do projeto

\chapter{Análise de Riscos}

A pesar de ser um software consideravelmente pequeno, quando falamos se sistemas coorporativos, ainda existem riscos que devem ser levados em consideração, mesmo que estes sejam completamente improváveis de acontecerem no curto-médio prazo. A seguir estão listados, em nível de probabilidade, e detalhados os riscos aqui levados em consideração.

\paragraph{Alta}

\paragraph{Média}

\paragraph{Baixa} Eventos de baixa probabilidade são aqueles que modificam estruturas que são externas à Defensoria Pública do Distrito Federal

\paragraph{Muito Baixa} Eventos de probabilidade baixa podem ser considerados catastróficos, ou seja, não só inviabilizam o desenvolvimento do software como impossibilitam o seu uso e existência da própria instituição. Entre eles estão:
\begin{itemize}
\item [-] Terremoto com epicentro próximo à Brasília devasta toda a estrutura jurídica de Brasília e todo o recurso é devidamente realocado
\item [-] Meteoro atinge o prédio da DP-DF
\item [-] Godzilla surge do fundo do Lago Paranoá e derruba o prédio da DP-DF
\item [-] ETs invadem a terra e sugam o prédio da DP-DF
\end{itemize}


Devido a baixa probabilidade de ocorrencia dos riscos de importância Muito Baixa, e são impossíveis de serem mitigados, estes serão descartados para fins deste projeto. 

%-> Discussão dos riscos do projeto e estratégia de mitigação
\bibliographystyle{plain}
\bibliography{es}

\end{document}
