% !TEX encoding = UTF-8 Unicode
\documentclass[12pt,a4paper]{report}
 
\usepackage[brazil]{babel}
\usepackage[utf8]{inputenc}
\usepackage[T1]{fontenc}
\usepackage{graphicx, subfigure}
\usepackage{epstopdf}
\usepackage{indentfirst}
\usepackage{fancyhdr}

\graphicspath{ {../images/} }

\epstopdfsetup{outdir=./}
\epstopdfsetup{suffix=}

\pagestyle{fancy}

\fancyhead[RO,RE]{\includegraphics{unb.eps}}
\fancyhead[LO,LE]{Departamento de Ciência da Computação\\
116441 --- Engenharia de Software}

\fancypagestyle{plain}{
\fancyhead[RO,RE]{\includegraphics{unb.eps}}
\fancyhead[LO,LE]{Departamento de Ciência da Computação\\
116441 --- Engenharia de Software}}

\title{Plano de Projeto do Sistema de Gestão Eletrônica de Documentos da Defensoria Pública do Distrito Federal - SGED/DPDF}
\author{Pedro Salum\\
	09/0139232\\
	pedro@loopec.com.br
	\and
	Daniel Sandoval\\
	09/0109899\\
	daniel@loopec.com.br}
	
\begin{document}
\maketitle
\tableofcontents

\chapter{Introdução}

Este documento tem por objetivo explicitar o planejamento para a realização do projeto de Gestão Eletrônica de Documentos da Defensoria Pública do Distrito Federal --- SGED/DPDF --- no tocante ao desenvolvimento, arquitetura, testes, implantação, manutenção e riscos. A seguir há uma breve descrição de cada capítulo.

\paragraph{Processo de Desenvolvimento} Discorre sobre decisões tomadas em relação a metodologias de desenvolvimento de \textit{software} e gerência de projeto. Discute a escolha de metodologias ágeis sobre métodos planejados.

\paragraph{Arquitetura e padrões de projeto} Discorre sobre escolhas de arquitetura e padrões de projeto que serão utilizados na codificação do sistema.

\paragraph{Plano de Testes} Discorre sobre as metodologias e ferramentas de testes escolhidas para o projeto, considerando deste testes de código até de aceitação, como e quando serão realizados.

\paragraph{Plano de Implantação e Manutenção} Discorre sobre as metodologias e ferramentas escolhidas para o processo de implantação e manutenção contínua do sistema entregue.

\paragraph{Análise de Riscos} Discorre sobre riscos que devem ser considerados ao longo do projeto e possibilidades de mitigação.

\chapter{Processo de Desenvolvimento}

%-> Definição e explicação de qual processo de desenvolvimento de software escolhido para o projeto

\chapter{Arquitetura e padrões de projeto}

%-> Definição e explicação da arquitetura de software escolhida para o projeto, focalizando a escolha de padrões de arquitetura, assim como diagrama de arquitetura
%-> Definição e explicação dos design patterns escolhidos para serem implementados no projeto

\chapter{Plano de Testes}
%-> Discussão dos testes de verificação e validação realizados no projeto, assim como descrição de casos de testes, inspeção de código, testes unitários

\chapter{Plano de Implantação e Manutenção}
%-> Discussão da estratégia de implantação e manutenção do projeto

\chapter{Análise de Riscos}
Enquanto a segurança, saúde e educação pública no DF piora, e piora, e piora, e piora,... [Yes, we have a large white elephant!] - http://copadomundo.uol.com.br/noticias/redacao/2013/03/03/obra-do-mane-garrincha-avanca-1-em-fevereiro-e-precisa-quintuplicar-ritmo-para-ficar-pronta-a-tempo.htm
%-> Discussão dos riscos do projeto e estratégia de mitigação

\end{document}
