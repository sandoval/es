% !TEX encoding = UTF-8 Unicode
\documentclass[12pt,a4paper]{report}
 
\usepackage[brazil]{babel}
\usepackage[utf8]{inputenc}
\usepackage[T1]{fontenc}
\usepackage{graphicx, subfigure}
\usepackage{epstopdf}
\usepackage{indentfirst}
\usepackage{fancyhdr}

\graphicspath{ {../images/} }

\epstopdfsetup{outdir=./}
\epstopdfsetup{suffix=}

\pagestyle{fancy}

\fancyhead[RO,RE]{\includegraphics{unb.eps}}
\fancyhead[LO,LE]{Departamento de Ciência da Computação\\
116441 --- Engenharia de Software}

\fancypagestyle{plain}{
\fancyhead[RO,RE]{\includegraphics{unb.eps}}
\fancyhead[LO,LE]{Departamento de Ciência da Computação\\
116441 --- Engenharia de Software}}

\title{Plano de Projeto do Sistema de Gestão Eletrônica de Documentos da Defensoria Pública do Distrito Federal - SGED/DPDF}
\author{Pedro Salum\\
	09/0139232\\
	pedro@loopec.com.br
	\and
	Daniel Sandoval\\
	09/0109899\\
	daniel@loopec.com.br}
	
\begin{document}
\maketitle
\tableofcontents

\chapter{Introdução}

Por outro lado, a hegemonia do ambiente político nos obriga à análise dos procedimentos normalmente adotados. Caros amigos, a contínua expansão de nossa atividade possibilita uma melhor visão global do orçamento setorial. Nunca é demais lembrar o peso e o significado destes problemas, uma vez que o desafiador cenário globalizado garante a contribuição de um grupo importante na determinação do processo de comunicação como um todo. A prática cotidiana prova que a crescente influência da mídia maximiza as possibilidades por conta das posturas dos órgãos dirigentes com relação às suas atribuições. 

Do mesmo modo, o julgamento imparcial das eventualidades estende o alcance e a importância das novas proposições. O incentivo ao avanço tecnológico, assim como a adoção de políticas descentralizadoras deve passar por modificações independentemente dos paradigmas corporativos. Desta maneira, a constante divulgação das informações facilita a criação das direções preferenciais no sentido do progresso. As experiências acumuladas demonstram que o aumento do diálogo entre os diferentes setores produtivos exige a precisão e a definição das condições inegavelmente apropriadas. 

Acima de tudo, é fundamental ressaltar que a consulta aos diversos militantes agrega valor ao estabelecimento dos índices pretendidos. É claro que o comprometimento entre as equipes acarreta um processo de reformulação e modernização das formas de ação. Podemos já vislumbrar o modo pelo qual o desenvolvimento contínuo de distintas formas de atuação pode nos levar a considerar a reestruturação dos modos de operação convencionais. Todas estas questões, devidamente ponderadas, levantam dúvidas sobre se o início da atividade geral de formação de atitudes não pode mais se dissociar dos relacionamentos verticais entre as hierarquias. 

Por conseguinte, a estrutura atual da organização oferece uma interessante oportunidade para verificação do sistema de participação geral. Percebemos, cada vez mais, que a necessidade de renovação processual ainda não demonstrou convincentemente que vai participar na mudança do sistema de formação de quadros que corresponde às necessidades. No mundo atual, o novo modelo estrutural aqui preconizado faz parte de um processo de gerenciamento de todos os recursos funcionais envolvidos. No entanto, não podemos esquecer que o entendimento das metas propostas apresenta tendências no sentido de aprovar a manutenção das diretrizes de desenvolvimento para o futuro. Gostaria de enfatizar que o acompanhamento das preferências de consumo estimula a padronização da gestão inovadora da qual fazemos parte. 

A nível organizacional, a consolidação das estruturas obstaculiza a apreciação da importância do remanejamento dos quadros funcionais. O cuidado em identificar pontos críticos na competitividade nas transações comerciais cumpre um papel essencial na formulação dos níveis de motivação departamental. Assim mesmo, o surgimento do comércio virtual afeta positivamente a correta previsão das condições financeiras e administrativas exigidas. Ainda assim, existem dúvidas a respeito de como o fenômeno da Internet representa uma abertura para a melhoria dos conhecimentos estratégicos para atingir a excelência.

\chapter{Processo de Desenvolvimento}

%-> Definição e explicação de qual processo de desenvolvimento de software escolhido para o projeto

\chapter{Arquitetura e padrões de projeto}

%-> Definição e explicação da arquitetura de software escolhida para o projeto, focalizando a escolha de padrões de arquitetura, assim como diagrama de arquitetura
%-> Definição e explicação dos design patterns escolhidos para serem implementados no projeto

\chapter{Plano de Testes}
%-> Discussão dos testes de verificação e validação realizados no projeto, assim como descrição de casos de testes, inspeção de código, testes unitários

\chapter{Plano de Implantação e Manutenção}
%-> Discussão da estratégia de implantação e manutenção do projeto

\chapter{Análise de Riscos}

A pesar de ser um software consideravelmente pequeno, quando falamos se sistemas coorporativos, ainda existem riscos que devem ser levados em consideração, mesmo que estes sejam completamente improváveis de acontecerem no curto-médio prazo. A seguir estão listados, em nível de probabilidade, e detalhados os riscos aqui levados em consideração.

\paragraph{Alta}

\paragraph{Média}

\paragraph{Baixa} Eventos de baixa probabilidade são aqueles que modificam estruturas que são externas à Defensoria Pública do Distrito Federal

\paragraph{Muito Baixa} Eventos de probabilidade baixa podem ser considerados catastróficos, ou seja, não só inviabilizam o desenvolvimento do software como impossibilitam o seu uso e existência da própria instituição. Entre eles estão:
\begin{itemize}
\item [-] Terremoto com epicentro próximo à Brasília devasta toda a estrutura jurídica de Brasília e todo o recurso é devidamente realocado
\item [-] Meteoro atinge o prédio da DP-DF
\item [-] Godzilla surge do fundo do Lago Paranoá e derruba o prédio da DP-DF
\item [-] ETs invadem a terra e sugam o prédio da DP-DF
\end{itemize}


Devido a baixa probabilidade de ocorrencia dos riscos de importância Muito Baixa, e são impossíveis de serem mitigados, estes serão descartados para fins deste projeto. 

%-> Discussão dos riscos do projeto e estratégia de mitigação

\end{document}
